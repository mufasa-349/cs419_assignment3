\documentclass[12pt,a4paper]{article}

\usepackage[T1]{fontenc}
\usepackage[utf8]{inputenc}
\usepackage[turkish]{babel}
\usepackage{geometry}
\usepackage{array}
\usepackage{tabularx}
\usepackage{multirow}
\usepackage{ragged2e}

\geometry{left=1.8cm,right=1.8cm,top=2.0cm,bottom=2.0cm}
\pagestyle{empty}

\setlength{\arrayrulewidth}{0.6pt}
\renewcommand{\arraystretch}{1.15}

% Likert (1..6) hücresi
\newcommand{\likertcell}{%
\begin{tabular}[c]{@{}c@{}}%
1\\2\\3\\4\\5\\6%
\end{tabular}%
}

\begin{document}

% ====== ÜST BİLGİ KUTULARI ======
\noindent
\begin{tabularx}{\textwidth}{|X|p{5.0cm}|}
\hline
\textbf{Hastanın Adı, Soyadı ve Yaşı:} \hfill\mbox{} &
\textbf{Tarih:}\hfill\mbox{} \\
\hline
\textbf{Başlama Tarihi:}\hspace{1.2cm} /\hspace{0.6cm}/\hspace{0.6cm}\hfill\mbox{} &
\textbf{Değerlendirdi:}\hfill\mbox{} \\
\hline
\end{tabularx}

\vspace{0.6cm}

% ====== BAŞLIK ======
\begin{center}
\textbf{\Large PREMENSTRÜEL DİSFORİDE SORUN \c{S}İDDETİ}\\
\textbf{\Large G\"UNL\"UK KAYIT \c{C}İZELGESİ (KISA FORM)}
\end{center}

\vspace{0.2cm}

\noindent
Her akşam aşağıda sıralanan sorunların şiddetlerini işaretleyiniz.
Sorunun şiddetini derecesine göre yuvarlak içine alınız.

\vspace{0.35cm}

% ====== SORUN ŞİDDET DERECELERİ KUTUSU ======
\begin{center}
\small
\begin{tabular}{|p{9.5cm}|}
\hline
\centering \textbf{Sorun Şiddet Dereceleri} \\
\hline
\begin{tabular}{p{4.6cm}p{4.6cm}}
1 = Hiç yok & 4 = Orta \\
2 = Çok hafif (minimal) & 5 = Şiddetli \\
3 = Hafif & 6 = Çok şiddetli \\
\end{tabular}
\\
\hline
\end{tabular}
\end{center}

\vspace{0.35cm}

% ====== ANA TABLO ======
\noindent\centering
\scriptsize
\begin{tabular}{|
>{\RaggedRight\arraybackslash}m{6.4cm}|
*{7}{>{\centering\arraybackslash}m{1.45cm}|}}
\hline
\textbf{Doğru günün altını işaretleyiniz} &
\textbf{Pzt.} & \textbf{Salı} & \textbf{Çar.} & \textbf{Per.} & \textbf{Cuma} & \textbf{C.tesi} & \textbf{Pazar} \\
\hline

Günün altına tarihi yazınız &
- & - & - & - & - & - & - \\
\hline

Lekelenme ya da adet kanamasını L ya da A ile kaydedin &
- & - & - & - & - & - & - \\
\hline

\textbf{1.} Kendimi üzgün, çökkün, huzursuz ya da umutsuz ya da değersiz ya da suçluymuşum gibi hissettim &
\likertcell & \likertcell & \likertcell & \likertcell & \likertcell & \likertcell & \likertcell \\
\hline

\textbf{2.} Kendimi sıkıntılı, endişeli ya da gergin hissettim &
\likertcell & \likertcell & \likertcell & \likertcell & \likertcell & \likertcell & \likertcell \\
\hline

\textbf{3.} Duygularımda ani değişiklikler (yani aniden üzülmek ve ağlamak) oldu ya da reddedilmeye çok duyarlıydım, çok kolay incindim &
\likertcell & \likertcell & \likertcell & \likertcell & \likertcell & \likertcell & \likertcell \\
\hline

\textbf{4.} Kendimi sinirli ya da kızgın hissettim. &
\likertcell & \likertcell & \likertcell & \likertcell & \likertcell & \likertcell & \likertcell \\
\hline

\textbf{5.} Olağan etkinliklerime (yani iş, okul, arkadaşlar, hobilerim, ev işleri) karşı ilgim azaldı &
\likertcell & \likertcell & \likertcell & \likertcell & \likertcell & \likertcell & \likertcell \\
\hline

\textbf{6.} Dikkatimi toplamada güçlük çektim &
\likertcell & \likertcell & \likertcell & \likertcell & \likertcell & \likertcell & \likertcell \\
\hline

\textbf{7.} Kendimi halsiz, yorgun ya da enerjisiz hissettim &
\likertcell & \likertcell & \likertcell & \likertcell & \likertcell & \likertcell & \likertcell \\
\hline
\end{tabular}

\vfill

% ====== FOOTER ======
\noindent\rule{\textwidth}{0.4pt}\par
\vspace{2mm}
\noindent\scriptsize
Premenstrual Disforide Sorun Şiddeti Günlük Kayıt Çizelgesi (Kısa Form)
\hfill
Sayfa 1 / 2

\end{document}
